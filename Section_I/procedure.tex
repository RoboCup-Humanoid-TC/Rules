\clearpage
\sffamily
{\bfseries\color[rgb]{0.4,0.4,0.4}
PROCEDURES TO DETERMINE THE WINNER OF A MATCH OR HOME-AND-AWAY }
\phantomsection
\addcontentsline{toc}{section}{Procedures to Determine The Winner of a Match or Home-And-Away}

\bigskip

Away goals, extra time, kicks from the penalty mark and extended kicks from the penalty mark are the four methods approved for determining the winning team where competition rules require there to be a winning team after a match has been drawn.

\bigskip

{\bfseries Away goals}

Competition rules may provide that where teams play each other home and away, if the aggregate score is equal after the second match, any goals scored at the ground of the opposing team will count double.

\bigskip

{\bfseries Extra time}

Competition rules may provide for two further equal periods, not exceeding 5 minutes each, to be played. The conditions of Law 8 will apply. 
\greyed{{(replaces: Competition rules may
provide for two further equal periods, not exceeding 15 minutes each, to be played. The conditions of Law 8 will apply. )}}

\bigskip

{\bfseries Kicks from the penalty mark }

Procedure

\headlinebox
 
\begin{itemize}
\item The referee chooses the goal at which the kicks will be taken
\item The referee tosses a coin and the team whose captain wins the toss decides whether to take the first or the second kick
\item The referee keeps a record of the kicks being taken 
\item Subject to the conditions explained below, both teams take five kicks 
\item The kicks are taken alternately by the teams 
\item If, before both teams have taken five kicks, one has scored more goals than the other could score, even if it were to complete its five kicks, no more kicks are taken
\item \greyed{(suspended: If, after both teams have taken five kicks, both have scored the same number of goals, or have not scored any goals, kicks continue to be taken in the same order until one team has scored a goal more than the other from the same number of kicks)}
\item A goalkeeper who is injured while kicks are being taken from the penalty mark and is unable to continue as goalkeeper may be replaced by a named substitute provided his team has not used the maximum number of substitutes permitted under the competition rules
\item With the exception of the foregoing case, only players who are on the field of play at the end of the match, which includes extra time where appropriate, (new) or which are suffering from a removal penalty or are currently in service, are eligible to take kicks from the penalty mark
\item \greyed{(suspended: Each kick is taken by a different player and all eligible players must take a kick before any player can take a second kick)}
\item An eligible player may change places with the goalkeeper at any time when kicks from the penalty mark are being taken
\item Only the eligible players and match officials are permitted to remain on the field of play when kicks from the penalty mark are being taken
\item All players, except the player taking the kick and the two goalkeepers, must remain within the centre circle
\item The goalkeeper who is the team-mate of the kicker must remain on the field of play, outside the penalty area in which the kicks are being taken, on the goal line where it meets the penalty area boundary line
\item Unless otherwise stated, the relevant Laws of the Game and International F.A. Board Decisions apply when kicks from the penalty mark are being taken
\item \removed{If at the end of the match and before the kicks start to be taken from the penalty mark, one team has a greater number of players than its opponents, it must reduce its numbers to equate with that of its opponents and the team captain must inform the referee of the name and number of each player excluded. Any player thus excluded may not participate in kicks from the penalty mark.}
\item Before the start of kicks from the penalty mark, the referee must ensure that an equal number of players from each team remains within the centre circle and they shall take the kicks
\end{itemize}


{\bfseries Extended kicks from the penalty mark (new) }

\bigskip

Procedure

\headlinebox

\begin{itemize}
\item All penalty shoots are taken on an empty goal.
\item The player performing the penalty kick may enter the goal area.
\item The team wins which...
\begin{enumerate}
\item ... kicked the ball into the goal / scores more often. If this is a tie:
\item ... kicked the ball into the goal area more often. If this is a tie:
\item ... touched the ball more often. If this is a tie:
\item ... in sum needed less time to score the goals. If this is a tie:
\item ... in sum needed less time to kick the ball into the goal area. If this is a tie:
\item ... in sum needed less time to touch the ball
\end{enumerate}
\item If this is a tie a coin is flipped
\end{itemize}
